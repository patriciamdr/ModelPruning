\documentclass[10pt,twocolumn,letterpaper]{article}
% \documentclass[12pt,letterpaper]{article}

\usepackage{listings}
\usepackage{cvpr}
\usepackage{times}
\usepackage{epsfig}
\usepackage{graphicx}
\usepackage{amsmath}
\usepackage{amssymb}
\usepackage{subfig}

% Include other packages here, before hyperref.

% If you comment hyperref and then uncomment it, you should delete
% egpaper.aux before re-running latex.  (Or just hit 'q' on the first latex
% run, let it finish, and you should be clear).
\usepackage[breaklinks=true,bookmarks=false]{hyperref}

\cvprfinalcopy % *** Uncomment this line for the final submission

\def\cvprPaperID{****} % *** Enter the CVPR Paper ID here
\def\httilde{\mbox{\tt\raisebox{-.5ex}{\symbol{126}}}}
\graphicspath{{./pics/}} % put all graphics here

% Pages are numbered in submission mode, and unnumbered in camera-ready
%\ifcvprfinal\pagestyle{empty}\fi
%\setcounter{page}{4321}
\begin{document}

%%%%%%%%% TITLE
% \title{\LaTeX\ Author Guidelines for CVPR Proceedings}
\title{Model Pruning: Weekly Report 4}
\author{Patricia Gschoßmann}

\maketitle
%\thispagestyle{empty}

%%%%%%%%% BODY TEXT
\section{Weekly Progress}
In this week selected experiments from last week were continued to improve the result.
As before, the goal was to prune 65\% of the parameters at once.
The following changes were applied compared to last week:
\begin{itemize}
	\item Initialize weights of convolutional layers in each parallel branch appropriately using He initialization\footnote{\url{https://openaccess.thecvf.com/content_iccv_2015/papers/He_Delving_Deep_into_ICCV_2015_paper.pdf}}
	\item Remove early stopping during $\alpha$-schedule
	\item Train for a maximum of 350 epochs for each $\alpha$
	\item Reduce the initial learning rate from 0.1 to 0.001
	\item Replace the previous learning rate scheduler; reduce the learning rate, whenever the validation loss stops improving\footnote{
			To clarify:
			The learning rate is reset to 0.001, whenever $\alpha$ drops, i.e.\ the learing rate schedule restarts.}
	\item Regarding regression: The loss now corresponds to the average MSE between each output value and its target (instead of the sum)
\end{itemize}

\section{Results}
\subsection{Classification}
In last week's experiments it was observed that the training for different $\alpha$-values stopped before the model reached its final performance, because of the early stopping callback.
Figure~\ref{fig:acc_classification} shows intermediate results for the new set up without this callback for different $\alpha$-values between $1$ and $0.4$.\footnote{The experiment is still running.}
The same $\alpha$-schedule as last week was used.
So far, the model maintains its original training accuracy.
Moreover, the model reached a higher validation accuracy for lower $\alpha$-values as expected due to a longer training period.
What is also interesting is, that model with lower $\alpha$-values sometimes outperform higher $\alpha$-values.\\
An improvement compared to last week is, that so far the model reached a validation accuracy higher than 90\% for each $\alpha$-value.
The experiment is still running, but based on this observation it is more likely, that the models maintains its performance until $\alpha=0$.
\begin{figure}[hpbt]
	\centering
	\subfloat[Training accuracy]{\includegraphics[scale=0.4]{pics/train_acc.png}}
	\hspace{0.1\textwidth}
	\subfloat[Validation accuracy]{\includegraphics[scale=0.4]{pics/val_acc_class.png}}
	\caption[]{Development of train- and validation accuracy during training for $\alpha=0.95$ (orange), $\alpha=0.6$ (red), $\alpha=0.45$ (light blue).
	An exponential schedule with a decay rate of $0.05$ was used.}
	\label{fig:acc_classification}
\end{figure}
\subsection{Regression}
Figure~\ref{fig:loss_regression} shows the training and validation loss for different $\alpha$ values between $1$ and $0.6$ for the new set up using regression.
The same $\alpha$-schedule was used as last week.
The development of the training loss clearly improved compared to last week:
It decreases continously for each $\alpha$.
It can also be seen that the loss would continue to decrease if the training units were extended to more than 350 epochs.
However, investigating the graph of the validation loss shows reveals extremly overfitting, which is why this experiment was aborted early.
% What is interesting is, that while the validation loss keeps increasing, the validation accuracy stagnates (see fig~\ref{fig:val_acc_regression}).
% One possible reason for that could be that the model is classifying the wrong classes with a higher confidence, i.e.\ the model increases its logits for the wrong classes.
% In this way, the total loss decreases, while the accuracy stays the same.
% This would indicate, that the loss, as it is implemented now, does not fit to the problem.
Further investigation is needed to discover the reason for this behavior.
Independently of that observation, an alternative approach would be to optimize over each logit error separately instead of optimizing based on their average error.
\begin{figure}[hpbt]
	\centering
	\subfloat[Training loss]{\includegraphics[scale=0.4]{pics/train_loss_regression.png}}
	\hspace{0.1\textwidth}
	\subfloat[Validation loss]{\includegraphics[scale=0.4]{pics/val_loss_regression.png}}
	\caption[]{Train- and validation loss during training as regression task for $\alpha$-values between $1$ and $0.6$.
	An exponential schedule with a decay rate of $0.05$ was used.}
	\label{fig:loss_regression}
\end{figure}
\begin{figure}[hpbt]
	\centering
	\includegraphics[scale=0.4]{pics/val_acc.png}
	\caption[]{Development of validation accuracy during training as regression task for $\alpha$-values.
	An exponential schedule with a decay rate of $0.05$ was used.}
	\label{fig:val_acc_regression}
\end{figure}

\section{Plan}
In the following week I continue with my current approach for the classification task, based on the (still open) results for the currently running experiment.
Moreover I want to investigate following approaches in order to improve my results:
\begin{itemize}
	\item Debug regression loss
	\item Allow weight updates for the pretrained model.
\end{itemize}

{\small
\bibliographystyle{ieee}
\bibliography{egbib}
}

\end{document}
