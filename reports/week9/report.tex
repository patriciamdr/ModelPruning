\documentclass[10pt,twocolumn,letterpaper]{article}

% \documentclass[12pt,letterpaper]{article}
\usepackage{listings}
\usepackage{titlesec}
\setcounter{secnumdepth}{4}
\usepackage{cvpr}
\usepackage{times}
\usepackage{epsfig}
\usepackage{graphicx}
\usepackage{amsmath}
\usepackage{amssymb}
\usepackage{subfig}

\titleformat{\paragraph}
{\normalfont\normalsize\bfseries}{\theparagraph}{1em}{}
\titlespacing*{\paragraph}
{0pt}{3.25ex plus 1ex minus .2ex}{1.5ex plus .2ex}

% Include other packages here, before hyperref.

% If you comment hyperref and then uncomment it, you should delete
% egpaper.aux before re-running latex.  (Or just hit 'q' on the first latex
% run, let it finish, and you should be clear).
\usepackage[breaklinks=true,bookmarks=false]{hyperref}

\cvprfinalcopy % *** Uncomment this line for the final submission

\def\cvprPaperID{****} % *** Enter the CVPR Paper ID here
\def\httilde{\mbox{\tt\raisebox{-.5ex}{\symbol{126}}}}
\graphicspath{{./pics/}} % put all graphics here

% Pages are numbered in submission mode, and unnumbered in camera-ready
%\ifcvprfinal\pagestyle{empty}\fi
%\setcounter{page}{4321}
\begin{document}

%%%%%%%%% TITLE
% \title{\LaTeX\ Author Guidelines for CVPR Proceedings}
\title{Model Pruning: Weekly Report 9}
\author{Patricia Gschoßmann}

\maketitle
%\thispagestyle{empty}

%%%%%%%%% BODY TEXT
\section{Weekly Progress}
In last week's experiments the $\alpha$-decay from 0.1 to 0.0 was too large to maintain the model's performance.
In this week the experiments were continued with several different approaches in order to improve the results:
\begin{itemize}
	\item Increase the learning rate at $\alpha=0$ (to 0.01 and 0.1)
	\item Restart pruning from $\alpha=0.1$ using an exponential $\alpha$-schedule with a decay-rate of 0.3 (and an initial learing rate of 0.001)
	\item Analyze the weights (i.e. "How big are they compared to the original model's weights?")
	\item Add weight decay of 1e-4
\end{itemize}
After multiple experiments, I came to the conclusion that I was not able to reduce the validation loss.
Unfortunately, I then discovered a bug in my \texttt{inference\_rgb.py} used to test the models:
The models' $\alpha$-values were not updated accordingly (i.e. $\alpha=0.1$ was used instead of $\alpha=0.0$), which is why the models' outputs looked not as expected.
The correct output of last week's model with $\alpha=0.0$ is depicted in fig.~\ref{fig:true}.
One can observe, that the output is a bit more blurred than in the original model output, however, the most prominent objects are still easily recognizable.
Unfortunately, this issue was only noticed at the end of this week's pruning process, which is why not many other results can be shown in this report.\\\\
Since last week's final model produced reasonable reconstructions, I tried to multiply the weights to shrink the model.
However, the implementation seems still to be incorrect as the smaller model only produces black images.
I suppose the mistake lies in the transposed convolution, as the shape of the weights is different to standard convolutions:
The dimensions for the in- and output channels are switched.
I am going to investigate this issue in the following week.
\begin{figure}[hpbt]
	\centering
	\subfloat[Original image\label{fig:original}]{\includegraphics[scale=0.175]{pics/rgb_original.png}}
	\qquad
	\subfloat[Original model output\label{fig:original_model}]{\includegraphics[scale=0.175]{pics/output.png}}
	\qquad
	\subfloat[Correctly reconstructed image data for last week's model at $\alpha=0$.]{\includegraphics[scale=0.175]{pics/0_output.png}}
	\caption[]{Original and reconstructed image data}
	\label{fig:true}
\end{figure}

\section{Addendum to report week 7: GPU test results}
As the GPU test results for the pruned VGG were rather unexpected, the implementation was revised.
It turned out, that I measured the wrong parameter (i.e., GPU utilization vs.\ used memory/free memory).
The following updated results were obtained:
On average, the original model took $\approx$0.055 seconds for each batch, while the pruned model took $\approx$0.053 seconds.\footnote{The duration measurement has not been changed, but is listed for the sake of completeness.}
The smaller model had an average unit consumption of 31.02\%, whereas the original model needed 41.41\% on average.
These results are much more in line with the expected impact of pruning.

\section{Plan}
\begin{itemize}
	\item Multiply weights correctly to obtain smaller model
	\item If successfull: Prune corresponding decoder based on smaller model
	\item Test if it is possible to prune encoder and decoder at the same time
\end{itemize}

{\small
\bibliographystyle{ieee}
\bibliography{egbib}
}

\end{document}
