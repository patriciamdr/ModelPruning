\documentclass[10pt,twocolumn,letterpaper]{article}
% \documentclass[12pt,letterpaper]{article}

\usepackage{listings}
\usepackage{titlesec}
\setcounter{secnumdepth}{4}
\usepackage{cvpr}
\usepackage{times}
\usepackage{epsfig}
\usepackage{graphicx}
\usepackage{amsmath}
\usepackage{amssymb}
\usepackage{subfig}

\titleformat{\paragraph}
{\normalfont\normalsize\bfseries}{\theparagraph}{1em}{}
\titlespacing*{\paragraph}
{0pt}{3.25ex plus 1ex minus .2ex}{1.5ex plus .2ex}

% Include other packages here, before hyperref.

% If you comment hyperref and then uncomment it, you should delete
% egpaper.aux before re-running latex.  (Or just hit 'q' on the first latex
% run, let it finish, and you should be clear).
\usepackage[breaklinks=true,bookmarks=false]{hyperref}

\cvprfinalcopy % *** Uncomment this line for the final submission

\def\cvprPaperID{****} % *** Enter the CVPR Paper ID here
\def\httilde{\mbox{\tt\raisebox{-.5ex}{\symbol{126}}}}
\graphicspath{{./pics/}} % put all graphics here

% Pages are numbered in submission mode, and unnumbered in camera-ready
%\ifcvprfinal\pagestyle{empty}\fi
%\setcounter{page}{4321}
\begin{document}

%%%%%%%%% TITLE
% \title{\LaTeX\ Author Guidelines for CVPR Proceedings}
\title{Model Pruning: Weekly Report 8}
\author{Patricia Gschoßmann}

\maketitle
%\thispagestyle{empty}

%%%%%%%%% BODY TEXT
\section{Weekly Progress}
In this week a new RGB autoencoder with a bottleneck of size $16\times16\times128$ was pruned to reduce 65\% of its encoder's parameters.
The original model's results can be seen in fig.~\ref{fig:original_model}.
\begin{figure}[hpbt]
	\centering
	\subfloat[Original image\label{fig:original}]{\includegraphics[scale=0.175]{pics/rgb_original.png}}
	\qquad
	\subfloat[Original model output\label{fig:original_model}]{\includegraphics[scale=0.175]{pics/output.png}}
	\caption[]{Original and reconstructed image data}
	\label{fig:results}
\end{figure}

\section{Results}
\subsection{Addendum to last week}
Last week's model reached a validation loss of 2.57 at $\alpha=0$.
The training was resumed repeatedly with different learning rates (0.01 and 0.1).
Unfortunately, this did not improve the performance.

\subsection{This week's results}
In contrast to last week, only the encoder was pruned in this week, as this affects the latent space.
The decoder then needs to be trained accordingly - this happens during the pruning process.
Moreover, each convolutional layer in the encoder was pruned, regardless of whether a transposed convolution follows\footnote{This only applies to the last convolution anyway}, leading to a reduced bottleneck.\\
The same setup as last week was used:
Pruning with weight updates of the pretrained model and a learning rate of 0.001 were used, as well as a step scheduler, which reduces $\alpha$ by 0.1 at each iteration.
The original model reached a minimum validation loss of 1.40.\\\\
For the validation development at each $\alpha$-decay the same behavior as last week could be observed:
Except for $\alpha=0$, the validation loss was the highest for the model with $\alpha=0.9$ with $val\_loss=2.12$.
With each $\alpha$-decay, the performance improved.\\
The model reached a validation loss of 2.05 at $\alpha=0.1$.
At the first iteration for $\alpha=0$ it only reached a validation loss of 2.53.
The training of the model was resumed with a learning rate of 0.01, however, the performance improved only slightly (2.48).
Corresponding validation loss curves can be seen in fig.~\ref{fig:curve}.
Intermediate outputs for $\alpha=0.1$ and $\alpha=0$ before pruning are shown in figure~\ref{fig:results}.
The differences are clearly visible.
However, the silhouettes of the objects in the image are somehow still visible.
\begin{figure}[hpbt]
	\centering
	\includegraphics[scale=0.4]{pics/val_loss.png}
	\caption{Validation loss development for $\alpha=0.1$ (red), $\alpha=0$ with $lr=0.001$ (blue) and the resumed training for $\alpha=0$ with $lr=0.01$ (green).}
	\label{fig:curve}
\end{figure}
\begin{figure}[hpbt]
	\centering
	\subfloat[Output at $\alpha=0.1$]{\includegraphics[scale=0.175]{pics/0_1.png}}
	\qquad
	\subfloat[Output at $\alpha=0$]{\includegraphics[scale=0.175]{pics/0.png}}
	\caption[]{Reconstructed image data during the pruning process. The original image is depicted in fig.~\ref{fig:original}.}
	\label{fig:results}
\end{figure}

\section{Plan}
\begin{itemize}
	\item Improve performance for $\alpha=0$ with learning rate approach (increase learning rate to 0.1) and smaller $\alpha$-decays
	\item Prune encoder correctly
	\item If successfull: Start pruning corresponding decoder
\end{itemize}

{\small
\bibliographystyle{ieee}
\bibliography{egbib}
}

\end{document}
