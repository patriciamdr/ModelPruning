\documentclass[10pt,twocolumn,letterpaper]{article}
% \documentclass[12pt,letterpaper]{article}

\usepackage{listings}
\usepackage{cvpr}
\usepackage{times}
\usepackage{epsfig}
\usepackage{graphicx}
\usepackage{amsmath}
\usepackage{amssymb}
\usepackage{subfig}

% Include other packages here, before hyperref.

% If you comment hyperref and then uncomment it, you should delete
% egpaper.aux before re-running latex.  (Or just hit 'q' on the first latex
% run, let it finish, and you should be clear).
\usepackage[breaklinks=true,bookmarks=false]{hyperref}

\cvprfinalcopy % *** Uncomment this line for the final submission

\def\cvprPaperID{****} % *** Enter the CVPR Paper ID here
\def\httilde{\mbox{\tt\raisebox{-.5ex}{\symbol{126}}}}
\graphicspath{{./pics/}} % put all graphics here

% Pages are numbered in submission mode, and unnumbered in camera-ready
%\ifcvprfinal\pagestyle{empty}\fi
%\setcounter{page}{4321}
\begin{document}

%%%%%%%%% TITLE
% \title{\LaTeX\ Author Guidelines for CVPR Proceedings}
\title{Model Pruning: Weekly Report 5}
\author{Patricia Gschoßmann}

\maketitle
%\thispagestyle{empty}

%%%%%%%%% BODY TEXT
\section{Weekly Progress}
In this week I continued working on the unfinished classification task from last week.
% The experiment was successfully completed, i.e.\ 65\% of the original model's parameters were pruned while maintaining an validation and accuracy of 90\%.
The experiment is nearly finished, i.e.\ the model currently reaches a validation accuracy of 85\% at $\alpha=0$.
% In total, the pruned model's accuracy dropped just 2\% compared to that of the original model.

\section{Results}
Last week's experiments were continued from $\alpha=0.4$.
Since an exponential scheduler with a decay rate of 0.05 was used until then, the following $\alpha$-decays would have been extremly small.
To speed up training, the maximum number of epochs was reduced to 100 and the decay rate was increased to 0.5 from this point on until $\alpha=0.018$.
Since all of these models maintained a stable performance, I decided to reduce the training duration even more by manually decreasing $\alpha$ to zero.
% Several attempts were executed, each based on a different previously trained model.
% However, all of them stagnated with a validation accuracy lower than 80\%.
% Figure~\ref{fig:attempts} shows the training and validation accuracy for selected attempts.
% Based on this observation, the initial learning rate was increased to overcome such local minima.
With a learning rate of $0.001$ the model was able to reach a validation accuracy of $\approx$ 73\%, based on the model, which was trained with $\alpha=0.05$ (see fig.~\ref{fig:final}, blue curve).
To overcome this local minima, the initial learning rate was increased - at the beginning to $0.01$ and then to $0.1$ - whenever the training stagnated.
In this way, a validation accuracy of $\approx$ 85\% was achieved.
Currently a further iteration of this approach is running, to overcome the 85\% accuracy bound.
Figure~\ref{fig:final} shows the corresponding training and validation accuracy of each iteration until now.
\begin{figure}[hpbt]
	\centering
	\centering
	% \subfloat[Training loss]{\includegraphics[scale=0.4]{pics/train_loss.png}}
	% \hspace{0.1\textwidth}
	\subfloat[Training accuracy]{\includegraphics[scale=0.4]{pics/train_acc.png}}
	\hspace{0.1\textwidth}
	% \subfloat[Validation loss]{\includegraphics[scale=0.4]{pics/val_loss.png}}
	% \hspace{0.1\textwidth}
	\subfloat[Validation accuracy]{\includegraphics[scale=0.4]{pics/val_acc.png}}
	\caption[]{Train- and validation accuracy for $\alpha=0.0$, based on the previously trained model with $\alpha=0.05$.
		Blue: First iteration with a learning rate of 0.001.
		Orange: Second iteration; first training was resumed with a lr of 0.01.
		Pink: Third iteration; resumed second training with a lr of 0.1.}
	\label{fig:final}
\end{figure}
% \subsection{Testing}
% After the final model reached the desired accuracy, its parallel branches were pruned to obtain the smaller version of it.
% The resulting pruned model was again tested on the test data, to ensure that the pruning was done correctly.

\section{Plan}
Once the model reached a validation accuracy of nearly 90\%, I am going to execute additonal tests to further compare the performances of the original and pruned model and determine other benefits besides the reduced storage space.
The tests focus on time and average memory consumption and are already implemented.\\\\
I additionally plan to determine a more suitable $\alpha$-schedule for the VGG16 on CIFAR10.
The fact, that I was able to change the exponential decay-rate during scheduling and even set $\alpha=0$ much sooner, than the specified scheduler would have, shows, that the used exponential decay main not be an ideal schedule for this task.
A schedule with larger $\alpha$-decays at the beginning and smaller at the end still makes sense, however, the decay should not become as small as with an exponential schedule.
Probably a multiplicative or time-based approach would be suitable.\\
The changes I applied to the learning rate additionally indicate, that a cosine annealing learning rate scheduler would help to solve the problem.
This type of lr scheduler was already used in previous experiments, but replaced by one, that reduces the learning rate once the validation loss stops decreasing.
The change was made, when the initial learning rate was decreased to prevent loss fluctuactions.\\\\
Furthermore, I aim to apply the pruning approach to other use-cases.

\section{Summary of new scripts}
\begin{itemize}
	\item \texttt{test.py}:
		Implements comparison between original and pruned model regarding time and memory consumption (GPU and CPU) on the test set.
\end{itemize}
{\small
\bibliographystyle{ieee}
\bibliography{egbib}
}

\end{document}
